\documentclass[english,notitlepage]{revtex4-1}  % defines the basic parameters of the document
%For preview: skriv i terminal: latexmk -pdf -pvc filnavn



% if you want a single-column, remove reprint

% allows special characters (including æøå)
\usepackage[utf8]{inputenc}
%\usepackage[english]{babel}

%% note that you may need to download some of these packages manually, it depends on your setup.
%% I recommend downloading TeXMaker, because it includes a large library of the most common packages.

\usepackage{physics,amssymb}  % mathematical symbols (physics imports amsmath)
\include{amsmath}
\usepackage{graphicx}         % include graphics such as plots
\usepackage{xcolor}           % set colors
\usepackage{hyperref}         % automagic cross-referencing (this is GODLIKE)
\usepackage{listings}         % display code
\usepackage[most]{tcolorbox}
\usepackage{inconsolata}
\usepackage{stmaryrd}
\usepackage{enumerate}
\usepackage{subfigure}        % imports a lot of cool and useful figure commands
\usepackage{float}
%\usepackage[section]{placeins}
\usepackage{algorithm}
\usepackage[noend]{algpseudocode}
\usepackage{subfigure}
\usepackage{tikz}
\usetikzlibrary{quantikz}
% defines the color of hyperref objects
% Blending two colors:  blue!80!black  =  80% blue and 20% black
% \hypersetup{ % this is just my personal choice, feel free to change things
%     colorlinks,
%     linkcolor={red!50!black},
%     citecolor={blue!50!black},
%     urlcolor={blue!80!black}}



\newcounter{tasknr}
\newcounter{subtasknr}[tasknr]
\renewcommand{\thetasknr}{\textbf{\arabic{tasknr}}}
\renewcommand{\thesubtasknr}{\textbf{(\alph{subtasknr})}}

\newenvironment{task}[1][\thetasknr]{\refstepcounter{tasknr}\par
\bigskip \noindent\textbf{Problem~\textbf{#1}\vspace{2em})}}
	{\par\bigskip}
\newenvironment{subtask}{\par\medskip \refstepcounter{subtasknr}
\noindent \textbf{(\alph{subtasknr})}}{\par\medskip}

\newcommand{\qed}{\hfill \ensuremath{\Box}}

\definecolor{codegreen}{rgb}{0,0.6,0}
\definecolor{codegray}{rgb}{0.5,0.5,0.5}
\definecolor{codepurple}{rgb}{0.58,0,0.82}
\definecolor{backcolour}{rgb}{0.95,0.95,0.92}

\lstdefinestyle{mystyle}{
    backgroundcolor=\color{backcolour},   
    commentstyle=\color{codegreen},
    keywordstyle=\color{magenta},
    numberstyle=\tiny\color{codegray},
    stringstyle=\color{codepurple},
    basicstyle=\ttfamily\footnotesize,
    breakatwhitespace=false,         
    breaklines=true,                 
    captionpos=b,                    
    % keepspaces=true,
    % numbers=left,
    % numbersep=5pt,
    showspaces=false,                
    showstringspaces=false,
    showtabs=false,
    tabsize=2
}
\lstset{style=mystyle}

%% Defines the style of the programming listing

%% This is actually my personal template, go ahead and change stuff if you want



%% USEFUL LINKS:
%%
%%   UiO LaTeX guides:        https://www.mn.uio.no/ifi/tjenester/it/hjelp/latex/
%%   mathematics:             https://en.wikibooks.org/wiki/LaTeX/Mathematics

%%   PHYSICS !                https://mirror.hmc.edu/ctan/macros/latex/contrib/physics/physics.pdf

%%   the basics of Tikz:       https://en.wikibooks.org/wiki/LaTeX/PGF/Tikz
%%   all the colors!:          https://en.wikibooks.org/wiki/LaTeX/Colors
%%   how to draw tables:       https://en.wikibooks.org/wiki/LaTeX/Tables
%%   code listing styles:      https://en.wikibooks.org/wiki/LaTeX/Source_Code_Listings
%%   \includegraphics          https://en.wikibooks.org/wiki/LaTeX/Importing_Graphics
%%   learn more about figures  https://en.wikibooks.org/wiki/LaTeX/Floats,_Figures_and_Captions
%%   automagic bibliography:   https://en.wikibooks.org/wiki/LaTeX/Bibliography_Management  (this one is kinda difficult the first time)
%%   REVTeX Guide:             http://www.physics.csbsju.edu/370/papers/Journal_Style_Manuals/auguide4-1.pdf
%%
%%   (this document is of class "revtex4-1", the REVTeX Guide explains how the class works)


%% CREATING THE .pdf FILE USING LINUX IN THE TERMINAL
%%
%% [terminal]$ pdflatex template.tex
%%
%% Run the command twice, always.
%% If you want to use \footnote, you need to run these commands (IN THIS SPECIFIC ORDER)
%%
%% [terminal]$ pdflatex template.tex
%% [terminal]$ bibtex template
%% [terminal]$ pdflatex template.tex
%% [terminal]$ pdflatex template.tex
%%
%% Don't ask me why; I don't know.

\begin{document}

\title{Project 2}      % self-explanatory
\author{Simon Halstensen \& Herman Brunborg}          % self-explanatory
\textsc{University of Oslo}
\date{\today}                             % self-explanatory
\noaffiliation                            % ignore this, but keep it.




\maketitle 

\url{https://github.com/sim-hal/FYS3150-Project-2}

\section*{Introduction}
    
In this artice we will mainly look at ways of scaling equations, some eigenvalue problems and unit testing.

\begin{task}

We want to show that the second-order differential equation 

\begin{equation}\label{eq:differentialequation}
    \gamma \frac{d^2 u(x)}{d x^2} = -F u(x)
\end{equation}

can be written as

\begin{equation}\label{eq:dimensionlessdifferentialequation}
    \frac{d^2 u(\hat{x})}{d \hat{x}^2} = -\lambda u(\hat{x})
\end{equation}

where $\hat{x}\equiv x/L \iff x$ is a dimensionless  variable and $\lambda=\frac{F L^2}{\gamma}$. This also means that $x = \hat{x} L$. We know that

\begin{equation*}
    \frac{du}{d\hat{x}} 
    = \frac{dx}{d\hat{x}} \frac{du}{dx} 
    = L \frac{du}{dx} 
    \iff \frac{du}{dx} 
    = \frac{1}{L} \frac{du}{d\hat{x}}
\end{equation*}

we define $v=\frac{1}{L} \frac{du}{d\hat{x}}$ and get

\begin{equation*}
    \frac{d^2 u}{d x^2}
    = \frac{d}{dx} \Big( \frac{d u}{d x} \Big)
    = \frac{d}{dx} \Big( \frac{1}{L}\frac{d u}{d \hat{x}} \Big) 
    = \frac{d}{dx} \Big( v \Big) 
    = \frac{d v}{dx}
\end{equation*}

\begin{equation*}
    = \frac{d\hat{x}}{dx} \frac{dv}{d\hat{x}}
    = \frac{1}{L} \frac{dv}{d\hat{x}}
    = \frac{1}{L} \frac{d}{d\hat{x}} \Big( \frac{1}{L}\frac{d u}{d \hat{x}}  \Big)
    = \frac{1}{L^2} \frac{d^2 u}{d\hat{x}^2} 
\end{equation*}

This means that

\begin{equation*}
    \gamma \frac{d^2 u}{d x^2} = \gamma \frac{1}{L^2} \frac{d^2 u}{d\hat{x}^2} = -F u
\end{equation*}

by moving terms we see that

\begin{equation*}
    \frac{d^2 u}{d\hat{x}^2} = -F L^2 \frac{1}{\gamma} u=-\lambda u
\end{equation*}

if we evaluate $u$ on $\hat{x}$ we see that

\begin{equation*}
    \frac{d^2 u(\hat{x})}{d\hat{x}^2} = -\lambda u(\hat{x})
\end{equation*}

\qed

\end{task}



\begin{task}

We know that $\vec{v}_i$ is a set of orthonormal basis vectors, meaning $\vec{v}_i^T\cdot \vec{v}_j = \delta_{ij}$. We also know that $\bold{U}$ is an orthonormal transformation matrix. This means that $\bold{U}^T=\bold{U}^{-1}$ and that $\bold{U} \bold{U}^T=\bold{U} \bold{U}^{-1}=I$. We want to show that transformations with $\bold{U}$ preserves orthonormality, and that the set of vectors $\vec{w}_i=\bold{U}\vec{v}_i$ is also an orthonormal set. We start b looking at $\vec{w}_i=\bold{U}\vec{v}_i$

\begin{equation*}
    \vec{w}_i = \bold{U} \vec{v}_i
\end{equation*}

we transpose this equation and get

\begin{equation*}
    \vec{w}_i^T = (\bold{U} \vec{v}_i)^T = \vec{v}_i^T \bold{U}^T
\end{equation*}

this means that

\begin{equation*}
    \vec{w}_i^T \vec{w}_j = \vec{v}_i^T \bold{U}^T \bold{U} \vec{v}_j = \vec{v}_i^T I \vec{v}_j = \vec{v}_i^T \vec{v}_j = \delta_{ij}
\end{equation*}

\qed


\end{task}


\end{document}
